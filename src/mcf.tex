\documentclass[UTF8,a4paper]{ctexart}
\usepackage{taoda}
\usepackage[lmargin=2cm,rmargin=2cm,tmargin=2cm,bmargin=2cm]{geometry}

\title{Notes on Minimal Cost Flow}
\author{陶大}

\begin{document}

\maketitle
\tableofcontents

\section{Definitions}

For a directed graph $G=(V, E)$, there is nonnegative capacity $u_e$ associated with each edge
and balance $b_v$ with each vertex.
For a flow over edge $e$, each unit costs $c_e$.

\[
    \min \sum_{e\in E} c_e x_e
\]
s.t.,
\begin{align}
    0\leqslant x_e \leqslant u_e,\qquad \forall e\in E\\
    \sum_{e\text{ from }v} x_e - \sum_{e\text{ into }v} x_e = b_v,\qquad \forall v\in V
\end{align}

Vertices with positive $b_v$ are called \emph{supplies},
and those with negative ones, \emph{demands}.
Solutions satisfy constraints are \emph{feasible}.

Without loss of generality, we can assume there is exactly one supply and one demand.
They are called the \emph{source} and the \emph{sink} respectively.

\begin{lemma}
    A maximal flow is also a feasible min-cost flow.
\end{lemma}

\begin{remark}
    Usually, we are interested in min-cost max-flow solutions.
    Let $G=(V, E, {u_e}, {c_e}, {b_v})$ be a MCF setting with a single source and a single sink,
    where a min-cost max flow is desired.
    Then we construct $G'=(V', E', {u'_e}, {c'_e}, {b'_v})$ as follows.
    \begin{align}
        V'&=V,\\
        E'&=E\cup\{e'\},\\
        u'_e&=\begin{cases}
            u_e,&\forall e\in E,\\
            +\infty,&e',
        \end{cases}\\
        c'_e&=\begin{cases}
            c_e,&\forall e\in E,\\
            -\infty,&e',
        \end{cases}\\
        b'_v&=\vec{0}
    \end{align}
    where $e'$ connects the only sink back to the source.
    The key is to minimize total cost, flow over $e'$ must be maximal.
    To achieve this, $c_{e'}$ must not be $-\infty$.
    $-\sum_{e\in E}\abs{c_e}$ is small enough.
\end{remark}

\section{Applications}


\end{document}
