\documentclass[UTF8,a4paper]{ctexart}
\usepackage[lmargin=2cm,rmargin=2cm,tmargin=2cm,bmargin=2cm]{geometry}
\usepackage{graphicx}
\usepackage{minted}
\usepackage{amsmath}
\usepackage{amssymb}
\usepackage{amsthm}
\usepackage{awesomebox}
\usepackage{hyperref}

\title{Notes on Minimal Cost Flow}
\author{陶大}

\renewcommand{\emph}[1]{\textbf{#1}}
\newcommand{\abs}[1]{\left| #1 \right|}

\begin{document}

\maketitle
\tableofcontents

\section{Definitions}

For a directed graph $G=(V, E)$, there is nonnegative capacity $u_e$ associated with each edge
and balance $b_v$ with each vertex.
For a flow over edge $e$, each unit costs $c_e$.

\[
    \min \sum_{e\in E} c_e x_e
\]
s.t.,
\begin{align}
    0\leqslant x_e \leqslant u_e,\qquad \forall e\in E\\
    \sum_{e\text{ from }v} x_e - \sum_{e\text{ into }v} x_e = b_v,\qquad \forall v\in V
\end{align}

Vertices with positive $b_v$ are called \emph{supplies},
and those with negative ones, \emph{demands}.
Solutions satisfy constraints are \emph{feasible}.

Without loss of generality, we can assume there is exactly one supply and one demand.
They are called the \emph{source} and the \emph{sink} respectively.

Usually, we are interested in min-cost solutions with maximal flows.
Let $G=(V, E, {u_e}, {c_e}, {b_v})$ be a MCF, which a min-cost max flow is desired.
Then we construct $G'=(V', E', {u'_e}, {c'_e}, {b'_v})$.
\begin{align}
    V'&=V,\\
    E'&=E\cup\{e'\},\\
    u'_e&=\begin{cases}
        u_e,&\forall e\in E,\\
        +\infinity,&e',
    \end{cases}\\
\end{align}

% \begin{lemma}
%     A maximal flow is also a feasible min-cost flow.
% \end{lemma}


\section{Applications}

\end{document}
